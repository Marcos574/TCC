\chapter[Introdução]{Introdução}


Em um mundo onde ocorre um rápido crescimento da competição de mercado, o produto que consegue oferecer algum diferencial para seu cliente  se torna cada vez mais almejado. Levando isso em consideração, o design de serviço ganha cada dia mais importância, pois o mesmo tem em seu objetivo principal, o foco no cliente. Nesse cenário, se faz necessária a melhor compreensão da prototipação no design de serviço, pois a mesma é um dos tópicos mais importantes nesse contexto.

A prototipação no design de serviços desempenha um papel fundamental ao possibilitar que as empresas testem, validem e refinem suas soluções antes de serem totalmente implementadas. Este processo não apenas oferece uma forma de visualizar as ideias de maneira concreta, mas também facilita a colaboração entre os stakeholders, permitindo que feedback valioso seja obtido de forma rápida e eficiente. Ter a contribuição dos usuários finais logo na fase de prototipagem assegura um software menos custoso e mais amigável ao usuário \cite{Mattjus2023}. Ao criar modelos e simulações das experiências planejadas, os protótipos possibilitam a detecção de falhas e a melhoria contínua das soluções, assegurando que o serviço final atenda às expectativas e necessidades do cliente de forma eficaz.

Portanto, a prototipação surge como uma etapa essencial no design de serviços. Através dela, é possível explorar diferentes abordagens, identificar pontos de melhoria e garantir que o serviço oferecido seja não apenas viável, mas também relevante e alinhado com as expectativas do público-alvo. Além disso, o estudo destaca que, embora os métodos de design de serviço sejam extremamente valiosos, esses métodos não são comumente conhecidos, e um guia sobre como utilizá-los é de grande importância \cite{Mattjus2023}. Portanto, entender profundamente o processo de prototipação é crucial para criar experiências de serviço que sejam não apenas inovadoras, mas também eficazes e impactantes no mercado.


\section {Problemas}

\section {Objetivos}
\subsection{Objetivo geral}

O objetivo da pesquisa é examinar o papel da prototipagem no Design de Serviço e entender melhor a mecânica por trás das técnicas e instrumentos usados.

Em suma, espera-se que os resultados desta pesquisa forneçam perspectivas sobre como o processo de prototipagem está presente no Design de Serviço, considerando e escolhendo técnicas e ferramentas adequadas.

\subsection{Objetivos específicos}

\begin{itemize}
	\item Explorar as principais técnicas de prototipação utilizadas no Design de Serviços, abordando suas características e aplicabilidade.
	
	\item Identificar as ferramentas de prototipação que são mais utilizadas no contexto do Design de Serviços, comparando as suas funcionalidades e adequação ao processo.
	
	\item Estabelecer a entrada e saída do processo de prototipação no Design de Serviços.%, detalhando sua interação com outras etapas do desenvolvimento.
\end{itemize}

\section{Metodologia}

Este estudo utiliza a metodologia de Revisão Sistemática de Literatura (RSL) para identificar e analisar as técnicas, ferramentas e processos de prototipação utilizados no Design de Serviços, assim como a importância do envolvimento do cliente nesse contexto. 

A Revisão sistemática de literatura (RSL) é uma forma de reconhecer, verificar e analisar questões ligadas ao tema de pesquisa. A revisão sistemática é definida como uma forma de estudo secundário, já que os estudos originais que são uma base para a mesma, são os primários.

A RSL tem como base três fases principais, sendo elas a fase de Planejamento, Condução e Resultados.

\begin{itemize}
	\item \textbf{Planejamento}
	
	Visa observar a necessidade dessa RSL. Nesse momento, é importante que o objetivo seja definido, o protocolo seja preparado, já que o mesmo servirá como um guia da RSL.
	
	\item \textbf{Condução}
	
	Tem como foco principal a identificação dos estudos, por meio da estratégia de busca que foi definida anteriormente na fase de planejamento. Os trabalhos escolhidos, são analisados, e o resultado desta análise são as respostas para as questões de pesquisa.
	
	
	\item \textbf{Resultados}
	
	Sendo a última fase da RSL, o objetivo desta fase é a documentação e descrição dos resultados, deixando desta maneira, as respostas preparadas para as questões de pesquisa.
\end{itemize}

\section{Estrutura da monografia}

Este trabalho está organizado em 4 capítulos, sendo eles

\begin{itemize}
	\item Introdução;
	
	\item Referencial teórico;
	
	\item Metodologia;
	
	\item Próximos passos;
\end{itemize}
