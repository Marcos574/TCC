\chapter[Introdução]{Introdução}



\section {Problemas}

\section {Objetivos}
\subsection{Objetivo geral}

O objetivo principal deste trabalho é analisar o papel da prototipação no contexto do Design de Serviços, investigando como as técnicas e ferramentas utilizadas contribuem para o desenvolvimento de soluções efetivas. Além disso, busca-se compreender os processos e a importância do envolvimento do cliente nesse contexto, com o intuito de reforçar a prototipação como uma etapa fundamental para a criação e validação de serviços.

Espera-se que os resultados desta pesquisa forneçam informações esclarecedoras sobre como a prototipação pode otimizar todo o processo de Design de Serviços, norteando a escolha de ferramentas e técnicas que são adequadas, além de promover práticas que reforcem a participação do cliente na construção de soluções mais eficientes e alinhadas às necessidades do usuário. 

\subsection{Objetivos específicos}

\begin{itemize}
	\item Explorar as principais técnicas de prototipação utilizadas no Design de Serviços, abordando suas características e aplicabilidade.
	
	\item Identificar as ferramentas de prototipação que são mais utilizadas no contexto do Design de Serviços, comparando as suas funcionalidades e adequação ao processo.
	
	\item Investigar como a prototipação ajuda no envolvimento do cliente em projetos de Design de Serviços.
	
	\item Estabelecer a entrada e saída do processo de prototipação no Design de Serviços, detalhando sua interação com outras etapas do desenvolvimento.
	conteúdo...
\end{itemize}

\section{Metodologia}

Este estudo utiliza a metodologia de Revisão Sistemática de Literatura (RSL) para identificar e analisar as técnicas, ferramentas e processos de prototipação utilizados no Design de Serviços, assim como a importância do envolvimento do cliente nesse contexto. 

A Revisão sistemática de literatura (RSL) é uma forma de reconhecer, verificar e analisar questões ligadas ao tema de pesquisa. A revisão sistemática é definida como uma forma de estudo secundário, já que os estudos originais que são uma base para a mesma, são os primários.

A RSL tem como base três fases principais, sendo elas a fase de Planejamento, Condução e Resultados.

\begin{itemize}
	\item \textbf{Planejamento}
	
	Visa observar a necessidade dessa RSL. Nesse momento, é importante que o objetivo seja definido, o protocolo seja preparado, já que o mesmo servirá como um guia da RSL.
	
	\item \textbf{Condução}
	
	Tem como foco principal a identificação dos estudos, por meio da estratégia de busca que foi definida anteriormente na fase de planejamento. Os trabalhos escolhidos, são analisados, e o resultado desta análise são as respostas para as questões de pesquisa.
	
	
	\item \textbf{Resultados}
	
	Sendo a última fase da RSL, o objetivo desta fase é a documentação e descrição dos resultados, deixando desta maneira, as respostas preparadas para as questões de pesquisa.
\end{itemize}

\section{Estrutura da monografia}

Este trabalho está organizado em 4 capítulos, sendo eles

\begin{itemize}
	\item Introdução;
	
	\item Referencial teórico;
	
	\item Metodologia;
	
	\item Próximos passos;
\end{itemize}
