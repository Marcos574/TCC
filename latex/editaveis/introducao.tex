\chapter[Introdução]{Introdução}


Atualmente, as empresas não competem apenas por meio de produtos, mas também pela qualidade dos serviços e das experiências proporcionadas aos seus clientes \cite{PineGilmore1999MuseumNews}. Com o avanço da transformação digital, disponibilizar serviços online já não é suficiente; é necessário repensar toda a experiência do usuário, considerando como ele interage com a organização ao longo de sua jornada. Projetar bons serviços, entretanto, envolve desafios específicos, pois eles são, em sua maioria, intangíveis e altamente dependentes da interação entre pessoas.

Nesse contexto, o Design de Serviços (DS) aparece como uma abordagem importante. O DS busca compreender profundamente as necessidades de todos os envolvidos (clientes, colaboradores, gestores, etc) para criar ou aprimorar serviços que sejam eficazes, agradáveis e coerentes com as expectativas dos usuários \cite{Stickdorn2019}. Inspirado em princípios do  \textit{design thinking}, o DS adota uma lógica colaborativa, multidisciplinar e iterativa, com o objetivo de orquestrar todos os pontos de contato e processos envolvidos na entrega de uma experiência de serviço coesa e de valor \cite{Polaine2013Orange, Mager2009Emerging}.

Dentro dessa abordagem, compreender profundamente a prototipação no DS é essencial, dado que é um dos elementos chave para o desenvolvimento de soluções. Tendo em vista que a prototipação no DS desempenha um papel fundamental ao possibilitar que as empresas testem, validem e refinem suas soluções antes da sua implementação definitiva \cite{paust2025integrative}.Esse processo não só concretiza ideias abstratas, transformando o intangível em tangível \cite{soto2023prototyping, mager2023product}, mas também promove a colaboração entre os \textit{stakeholders} \cite{paust2025integrative}, possibilitando a coleta de feedback valioso de maneira ágil e eficiente \cite{asbjornsen2022echange}.Ao incorporar a perspectiva dos usuários finais desde a prototipagem, assegura um software (ou serviço) com redução de custos desnecessários de desenvolvimento e garante-se um produto mais intuitivo e alinhado às necessidades reais \cite{Mattjus2023, villa2022integratedcare}.

Além disso, por meio de simulações e encenações das experiências planejadas — como \textit{role-playing} ou \textit{walkthroughs} \cite{seko2024transitions, soto2023prototyping} —, a prototipação permite detectar falhas, testar alternativas e promover melhorias contínuas. Dessa forma, ela se configura como uma etapa indispensável no DS, viabilizando a exploração de diferentes abordagens \cite{paust2025integrative}, a identificação de oportunidades de aperfeiçoamento e a garantia de que o serviço final atenda, de maneira eficaz, às necessidades e expectativas do público-alvo \cite{kumar2023rheumatology}.


\section {Problemas}

A transformação é uma área em constante ascensão atualmente. A simples digitalização de serviços não é mais suficiente, devendo os serviços serem oferecidos como uma solução digital que preze não apenas pelo seu funcionamento em si, mas também pela experiência do usuário e sua jornada ao longo da execução do serviço. Nesse contexto, o Design de Serviços surge como uma ferramenta aplicada na elaboração dessa transformação digital, procurando envolver não apenas os \textit{stakeholders}, mas também os usuários, desde a fase de ideação. A prototipação, que é uma ferramenta usada na Engenharia de Software, surge como uma opção forte no Design de Serviços para mostrar soluções digitais antes de sua implementação, bem como conduzir diversos testes, incluindo inclusive o usuário no processo. Desta forma, a aplicação da prototipação no processo de Design de Serviços surge como uma novidade ainda pouco explorada. Neste sentido, entender como a prototipação pode ser explorada para corroborar com o processo de DS é um problema relevante.

\section {Objetivos}
\subsection{Objetivo geral}

O objetivo geral desta pesquisa é realizar uma Revisão Sistemática da Literatura para sintetizar o conhecimento atual sobre a aplicação da prototipagem no Design de Serviços, examinando as técnicas, caracterizações, ferramentas, contextos de aplicação, comparações e processos de entrada e saída envolvidos. % A segunda frase pode ser movida para a conclusão ou integrada de forma diferente.

\subsection{Objetivos específicos}

\begin{itemize}
	\item Levantar as técnicas de prototipação utilizadas no DS, abordando suas características e aplicabilidade.
	
	\item Identificar as ferramentas de prototipação que são mais utilizadas no contexto do design de serviços, comparando as suas funcionalidades e adequação ao processo.
	
	\item Estabelecer a entrada e a saída de um processo de prototipação no DS padrão.%, detalhando sua interação com outras etapas do desenvolvimento.
\end{itemize}

\section{Metodologia}

Este estudo utiliza a metodologia de Revisão Sistemática de Literatura (RSL) para identificar e analisar as técnicas, ferramentas e processos de prototipação utilizados no Design de Serviços (DS), assim como a importância do envolvimento dos clientes nesse contexto. 

A Revisão Sistemática de Literatura (RSL) é uma forma de reconhecer, verificar e analisar questões ligadas ao tema de pesquisa. Ela é definida como uma forma de estudo secundário, já que os estudos originais que são uma base para a mesma, são os primários. 

A RSL tem como base três fases principais, sendo elas a fase de Planejamento, Condução e Resultados. 

\begin{itemize} 
	\item \textbf{Planejamento:} Visa observar a necessidade desta Revisão Sistemática de Literatura (RSL). Nesse momento, é importante que o objetivo seja definido e o protocolo preparado, já que este servirá como um guia para a RSL. 
	
	\item \textbf{Condução:} Tem como foco principal a identificação dos estudos, por meio da estratégia de busca definida anteriormente na fase de planejamento. Os trabalhos escolhidos são analisados, e os resultados dessa análise fornecem as respostas para as questões de pesquisa. 
	
	\item \textbf{Resultados:} Sendo a última fase da RSL, o objetivo desta fase é a documentação e descrição dos resultados, deixando assim as respostas preparadas para as questões de pesquisa.
	
\end{itemize}


\section{Estrutura da monografia}

Este trabalho está organizado em 5 capítulos, sendo eles:

\begin{itemize}
	\item Introdução;
	
	\item Referencial teórico;
	
	\item Metodologia;
	
	\item Resultados;
	
	\item Conclusão;
\end{itemize}
