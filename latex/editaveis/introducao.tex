\chapter[Introdução]{Introdução}


Em um mundo de crescente competitividade de mercado, os produtos e serviços que oferecem diferenciais significativos tornam-se cada vez mais desejados pelos clientes. Nesse contexto, o Design de Serviços(DS) ganha relevância, pois seu objetivo principal é colocar o cliente no centro das decisões. Dentro desse cenário, compreender profundamente a prototipação no DS é essencial, já que ela é um dos elementos mais importantes para o desenvolvimento de soluções eficazes e inovadoras.

A prototipação no DS desempenha um papel fundamental ao possibilitar que as empresas testem, validem e refinem suas soluções antes de serem totalmente implementadas. Esse processo não apenas oferece uma forma de visualizar as ideias de maneira concreta, mas também facilita a colaboração entre os \textit{stakeholders}, possibilitando a coleta de feedback valioso de maneira ágil e eficiente. Incorporar a perspectiva dos usuários finais logo na fase de prototipagem assegura um software menos custoso e mais amigável ao usuário \cite{Mattjus2023}. Por meio de modelos e simulações das experiências planejadas, os protótipos possibilitam a detecção de falhas e a melhoria contínua das soluções, assegurando que o serviço final atenda às expectativas e necessidades do cliente de forma eficaz.

Portanto, a prototipação surge como uma etapa essencial no DS: através dela, é possível explorar diferentes abordagens, identificar pontos de melhoria e garantir que o serviço oferecido seja não apenas viável, mas também relevante e alinhado com as expectativas do público-alvo.  No entanto, é importante destacar que, apesar do valor significativo dos métodos de Design de Serviços, eles ainda são pouco conhecidos. Assim, um guia sobre sua aplicação torna-se uma ferramenta de grande importância \cite{Mattjus2023}. Assim sendo, entender o processo de prototipação é crucial para criar experiências de serviço que sejam não apenas inovadoras, mas também eficazes e impactantes no mercado.


\section {Problemas}

A transformação é uma área em constante ascensão atualmente. A simples digitalização de serviços não é mais suficiente, devendo os serviços serem oferecidos como uma solução digital que preze não apenas pelo seu funcionamento em si, mas também pela experiência do usuário e sua jornada ao longo da execução do serviço. Nesse contexto, o Design de Serviços surge como uma ferramenta aplicada na elaboração dessa transformação digital, procurando envolver não apenas os \textit{stakeholders}, mas também os usuários, desde a fase de ideação. A prototipação, que é uma ferramenta amplamente usada na Engenharia de Software, surge como uma opção forte no Design de Serviços para mostrar a solução digital antes de sua implementação, bem como conduzir diversos testes, incluindo inclusive o usuário no processo. Desta forma, a aplicação da prototipação no processo de Design de Serviços surge como uma novidade ainda pouco explorada. Neste sentido, entender como a prototipação pode ser explorada para corroborar com o processo de DS é um problema relevante.

\section {Objetivos}
\subsection{Objetivo geral}

O objetivo principal desta pesquisa é examinar o papel da prototipagem no design de serviço e entender melhor a mecânica por trás das técnicas e instrumentos usados.

Em suma, espera-se que os resultados desta pesquisa forneçam perspectivas sobre como o processo de prototipagem está presente no design de serviço, considerando e escolhendo técnicas e ferramentas adequadas.

\subsection{Objetivos específicos}

\begin{itemize}
	\item Explorar as principais técnicas de prototipação utilizadas no DS, abordando suas características e aplicabilidade.
	
	\item Identificar as ferramentas de prototipação que são mais utilizadas no contexto do design de serviços, comparando as suas funcionalidades e adequação ao processo.
	
	\item Estabelecer a entrada e saída do processo de prototipação no DS.%, detalhando sua interação com outras etapas do desenvolvimento.
\end{itemize}

\section{Metodologia}

Este estudo utiliza uma versão abreviada da metodologia de Revisão Sistemática de Literatura (RSL) para identificar e analisar as técnicas, ferramentas e processos de prototipação utilizados no Design de Serviços (DS), assim como a importância do envolvimento dos clientes nesse contexto. Optou-se por uma versão reduzida da RSL devido às limitações de tempo do Trabalho de Conclusão de Curso (TCC), uma vez que a aplicação completa da metodologia exigiria um período mais extenso para sua execução. 

A Revisão Sistemática de Literatura (RSL) é uma forma de reconhecer, verificar e analisar questões ligadas ao tema de pesquisa. Ela é definida como uma forma de estudo secundário, já que os estudos originais que são uma base para a mesma, são os primários.

A RSL tem como base três fases principais, sendo elas a fase de Planejamento, Condução e Resultados.

\begin{itemize}
	\item \textbf{Planejamento}
	
	Visa observar a necessidade desta Revisão Sistemática de Literatura (RSL). Nesse momento, é importante que o objetivo seja definido e o protocolo preparado, já que este servirá como um guia para a RSL.
	
	\item \textbf{Condução}
	
	Tem como foco principal a identificação dos estudos, por meio da estratégia de busca definida anteriormente na fase de planejamento. Os trabalhos escolhidos são analisados, e os resultados dessa análise fornecem as respostas para as questões de pesquisa.
	
	\item \textbf{Resultados}
	
	Sendo a última fase da RSL, o objetivo desta fase é a documentação e descrição dos resultados, deixando assim as respostas preparadas para as questões de pesquisa.
\end{itemize}

\section{Estrutura da monografia}

Este trabalho está organizado em 4 capítulos, sendo eles:

\begin{itemize}
	\item Introdução;
	
	\item Referencial teórico;
	
	\item Metodologia;
	
	\item Próximos passos;
\end{itemize}
