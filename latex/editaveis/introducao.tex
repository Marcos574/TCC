\chapter[Introdução]{Introdução}


Hoje em dia, as empresas competem não só com produtos, mas também com a qualidade dos serviços e das experiências que oferecem aos clientes \cite{PineGilmore1999MuseumNews}. Com a transformação digital, não basta apenas colocar serviços online; é preciso redesenhar tudo pensando na jornada do usuário e em como ele interage com a empresa. Criar bons serviços, no entanto, apresenta desafios próprios, pois eles são muitas vezes intangíveis (não podemos pegar) e dependem muito da interação entre pessoas.

Nesse cenário, o Design de Serviços (DS) aparece como uma abordagem importante. O DS busca entender profundamente as necessidades das pessoas (clientes, funcionários, etc.) para criar ou melhorar serviços, fazendo com que funcionem bem e sejam agradáveis de usar \cite{Stickdorn2019}. Ele usa uma forma de pensar criativa e centrada no ser humano, muitas vezes inspirada no \textit{design thinking}, e envolve a colaboração de diversas pessoas no processo. O objetivo é organizar todos os pontos de contato e processos para entregar uma experiência de serviço completa e de valor \cite{Polaine2013Orange, Mager2009Emerging}.

Dentro desse cenário, compreender profundamente a prototipação no DS é essencial, já que ela é um dos elementos mais importantes para o desenvolvimento de soluções. A prototipação no DS desempenha um papel fundamental ao possibilitar que as empresas testem, validem e refinem suas soluções antes de serem totalmente implementadas \cite{paust2025integrative}. Esse processo não apenas oferece uma forma de visualizar as ideias de maneira concreta (tornando o intangível, tangível \cite{soto2023prototyping, mager2023product}), mas também facilita a colaboração entre os \textit{stakeholders} \cite{paust2025integrative}, possibilitando a coleta de feedback valioso de maneira ágil e eficiente \cite{asbjornsen2022echange}. Incorporar a perspectiva dos usuários finais logo na fase de prototipagem assegura um software (ou serviço) menos custoso e mais amigável ao usuário \cite{Mattjus2023, villa2022integratedcare}. Por meio de modelos e simulações das experiências planejadas (como \textit{role-playing} ou \textit{walkthroughs} \cite{seko2024transitions, soto2023prototyping}), os protótipos possibilitam a detecção de falhas e a melhoria contínua das soluções, assegurando que o serviço final atenda às expectativas e necessidades do cliente de forma eficaz \cite{kumar2023rheumatology}. Portanto, a prototipação surge como uma etapa essencial no DS: através dela, é possível explorar diferentes abordagens \cite{paust2025integrative}, identificar pontos de melhoria e garantir que o serviço oferecido seja não apenas viável, mas também relevante e alinhado com as expectativas do público-alvo.

\section {Problemas}

A transformação é uma área em constante ascensão atualmente. A simples digitalização de serviços não é mais suficiente, devendo os serviços serem oferecidos como uma solução digital que preze não apenas pelo seu funcionamento em si, mas também pela experiência do usuário e sua jornada ao longo da execução do serviço. Nesse contexto, o Design de Serviços surge como uma ferramenta aplicada na elaboração dessa transformação digital, procurando envolver não apenas os \textit{stakeholders}, mas também os usuários, desde a fase de ideação. A prototipação, que é uma ferramenta amplamente usada na Engenharia de Software, surge como uma opção forte no Design de Serviços para mostrar a solução digital antes de sua implementação, bem como conduzir diversos testes, incluindo inclusive o usuário no processo. Desta forma, a aplicação da prototipação no processo de Design de Serviços surge como uma novidade ainda pouco explorada. Neste sentido, entender como a prototipação pode ser explorada para corroborar com o processo de DS é um problema relevante.

\section {Objetivos}
\subsection{Objetivo geral}

O objetivo geral desta pesquisa é realizar uma Revisão Sistemática da Literatura para sintetizar o conhecimento atual sobre a aplicação da prototipagem no Design de Serviços, examinando as técnicas, caracterizações, ferramentas, contextos de aplicação, comparações e processos de entrada e saída envolvidos. % A segunda frase pode ser movida para a conclusão ou integrada de forma diferente.

\subsection{Objetivos específicos}

\begin{itemize}
	\item Levantar as principais técnicas de prototipação utilizadas no DS, abordando suas características e aplicabilidade.
	
	\item Identificar as ferramentas de prototipação que são mais utilizadas no contexto do design de serviços, comparando as suas funcionalidades e adequação ao processo.
	
	\item Estabelecer a entrada e a saída de um processo de prototipação no DS padrão.%, detalhando sua interação com outras etapas do desenvolvimento.
\end{itemize}

\section{Metodologia}

Este estudo utiliza a metodologia de Revisão Sistemática de Literatura (RSL) para identificar e analisar as técnicas, ferramentas e processos de prototipação utilizados no Design de Serviços (DS), assim como a importância do envolvimento dos clientes nesse contexto. 

A Revisão Sistemática de Literatura (RSL) é uma forma de reconhecer, verificar e analisar questões ligadas ao tema de pesquisa. Ela é definida como uma forma de estudo secundário, já que os estudos originais que são uma base para a mesma, são os primários. 

A RSL tem como base três fases principais, sendo elas a fase de Planejamento, Condução e Resultados. 

\begin{itemize} 
	\item \textbf{Planejamento:} Visa observar a necessidade desta Revisão Sistemática de Literatura (RSL). Nesse momento, é importante que o objetivo seja definido e o protocolo preparado, já que este servirá como um guia para a RSL. 
	
	\item \textbf{Condução:} Tem como foco principal a identificação dos estudos, por meio da estratégia de busca definida anteriormente na fase de planejamento. Os trabalhos escolhidos são analisados, e os resultados dessa análise fornecem as respostas para as questões de pesquisa. 
	
	\item \textbf{Resultados:} Sendo a última fase da RSL, o objetivo desta fase é a documentação e descrição dos resultados, deixando assim as respostas preparadas para as questões de pesquisa.
	
\end{itemize}


\section{Estrutura da monografia}

Este trabalho está organizado em 5 capítulos, sendo eles:

\begin{itemize}
	\item Introdução;
	
	\item Referencial teórico;
	
	\item Metodologia;
	
	\item Resultados;
	
	\item Conclusão;
\end{itemize}
