\chapter[Metodologia]{Metodologia}

Este capítulo apresenta a metodologia utilizada, que se trata de como o trabalho foi desenvolvido, visando responder as questões elaboradas na introdução.

A metodologia selecionada foi a Revisão Sistemática de Literatura (RSL). Se trata de uma abordagem rigorosa para identificar, avaliar e sintetizar evidências relevantes sobre uma questão específica de pesquisa.
É um método amplamente utilizado em muitas áreas para fixar o entendimento sobre algum tema em específico e direcionar pesquisas futuras. A RSL se trata de um estudo secundário, pois é baseado em estudos primários que já foram publicados.

Para conduzir uma RSL de maneira estruturada e eficiente, o processo é dividido em três fases principais, o \textbf{planejamento}, a \textbf{condução} e os \textbf{resultados}.

\begin{itemize}
	\item \textbf{Planejamento}

A fase de planejamento é crucial para a organização da RSL. O objetivo desta etapa é identificar a necessidade de realizar a revisão e definir de maneira clara os objetivos. Isso inclui duas atividades-chave, sendo elas:

\begin{itemize}
	\item \textbf{Formular as questões de pesquisa}
	
		Funcionam como um guia para o processo de busca e análise.
	
	\item \textbf{Elaboração do protoloco}
	
		 Descreve os critérios de inclusão e exclusão de estudos. Além disso, define quais serão as fontes de dados a serem utilizadas (como bases de dados eletrônicas, livros, anais de conferências e outras publicações relevantes). Por último, mas não menos importante, os procedimentos para coleta e organização das informações. Esse protocolo serve como um guia para toda a RSL.
\end{itemize}

\item \textbf{Condução}

Esta etapa é a fase operativa da RSL, onde a estratégia de busca definida anteriormente é colocada de fato em prática. Essa estratégia pode incluir combinações de palavras-chave, operadores lógicos e filtros específicos que permitam localizar os estudos mais relevantes para a pesquisa nas fontes selecionadas. Uma vez coletados, os estudos passam por uma triagem baseada nos critérios de inclusão e exclusão estabelecidos no protocolo, para assim, filtrar ainda mais os estudos.

Logo após, é realizada uma análise crítica do conteúdo dos estudos selecionados, com o objetivo de extrair e organizar as informações relevantes. A principal finalidade dessa fase é responder às questões de pesquisa propostas, garantindo que a resposta tenha como base as evidências obtidas a partir dos estudos.


\item \textbf{Resultado}

Esta é a fase final da RSL, que é dedicada à documentação e apresentação dos resultados. Aqui, os dados extraídos e analisados são consolidados em um relatório ou artigo científico, que inclui a descrição detalhada dos procedimentos adotados, as evidências encontradas e as respostas às questões de pesquisa. Esta fase também pode envolver a elaboração de tabelas, gráficos e diagramas que auxiliem na compreensão dos resultados.

É importante destacar que os achados da RSL devem ser interpretados considerando-se suas limitações, como possíveis vieses nos estudos primários ou na seleção dos mesmos.

\end{itemize}

\section{Planejamento}

\subsection{Questões de pesquisa}

Para elaboração das questões de pesquisa, os objetivos desse trabalho foram utilizados como base principal. As questões 1, 2 e 3 estão ligadas ao primeiro objetivo específico, as questões 4 e 5 estão ligadas ao segundo objetivo específico e a questão 6 está ligada ao terceiro objetivo específico. Como resultado, estão listadas abaixo as questões de pesquisa.

%Talvez fazer uma tabela relacionando cada questão com seu objetivo específico relativo

\begin{enumerate}
	
	\item Quais são as principais técnicas de prototipação utilizadas no Design de Serviços?
	\item Como essas técnicas são caracterizadas em termos de objetivos, métodos e resultados esperados?
	\item Em quais contextos ou etapas do Design de Serviços essas técnicas são mais aplicáveis?
	
	\item Quais ferramentas de prototipação são mais utilizadas no Design de Serviços?
	\item Como essas ferramentas se comparam em termos de funcionalidades e adequação às diferentes etapas do processo de Design de Serviços?
	
	\item Quais são as entradas necessárias e as saídas esperadas no processo de prototipação em Design de Serviços?
	
\end{enumerate}


\subsection{Estratégia de busca}

A estratégia de busca dessa RSL foi elaborada utilizando como base o modelo PICOC (População, Intervenção, Comparação, Resultado e Contexto). Este modelo fornece uma estrutura lógica para a definição de critérios de busca. Logo abaixo, cada elemento do PICOC é descrito em relação ao tema do trabalho.

\begin{itemize}
	\item \textbf{População (P)}
	
	Define o grupo de interesse relacionado ao tema do estudo.
	
	Neste estudo, a população abrange estudos relacionados ao Design de Serviços.
	\begin{itemize}
		\item Palavras-chave: "Service Design", "Service Development".
	\end{itemize}
	
	\item \textbf{Intervenção (I)}
	
	Estabelece as ações, técnicas ou ferramentas que estão sendo investigadas.
	
	Neste estudo, a intervenção está relacionada à \textbf{prototipação}, incluindo técnicas e ferramentas utilizadas nesse contexto.
	
	\begin{itemize}
		\item Palavras-chave: "Prototyping", "Prototyping Techniques", "Prototyping Tools".
	\end{itemize}
	
	\item \textbf{Comparação (C)}
	
	Especifica se haverá alguma comparação explícita entre alternativas.
	
	Neste estudo não haverá nenhuma comparação obrigatória, porém alguns estudos podem acabar comparando diferentes técnicas ou ferramentas de prototipação. 
	
	\item \textbf{Resultado (O)}
	
	Indica os resultados esperados.
	
	Neste estudo, os resultados esperados incluem identificação de ferramentas de prototipação, informações sobre entradas e saídas do processo e sua interação com outras etapas do desenvolvimento.
	
	\begin{itemize}
		\item Palavras-chave: "Effectiveness", "Outcomes", "Results", "Impact".
	\end{itemize} 
	
	\item \textbf{Contexto (C)}
	%Repensar o contexto, talvez usar o contexto de prototipação para resultados mais abrangentes.
	
	Estabelece o cenário em que o estudo foi realizado.
	Neste estudo, o contexto inclui estudos que abordem sobre prototipação em geral além do design de serviços. Os estudos selecionados devem obrigatoriamente abordar ao menos um dos 2 temas.
	
	\begin{itemize}
		\item Palavras-chave: "Prototyping Process", "Prototyping Applications", "Service Design", "Service Prototyping", "Service Development".
	\end{itemize}
	
\end{itemize}

\subsection{String de busca}

Com base nos elementos do modelo PICOC, principalmente nas palavras-chave, uma string de busca foi desenvolvida utilizando operadores booleanos (AND, OR) para combinar os termos identificados e realizar a busca pelos estudos nas bases de dados definidas. A string foi adaptada para cada base de dados utilizada, visando respeitas suas especificidades. Logo abaixo a string base é apresentada:


\textbf{(("prototyping" OR "prototyping techniques" OR "prototyping tools" OR "prototyping process") AND 
("process" OR "methods" OR "outcomes" OR "results")) 
\newline      OR \newline 
(("service design" OR "service prototyping" OR "service development") AND 
("prototyping" OR "prototyping tools" OR "prototyping techniques"))}

\subsection{Bases de dados}

Para escolha das bases de dados, alguns critérios foram levados em consideração, como relevância da base e principalmente a possibilidade de acesso à mesma através da Universidade de Brasília. Como resultado, as bases selecionadas foram a \textbf{Scopus} e \textbf{CAPES}.

\subsection{Critérios de seleção}

Para realizar uma triagem primária, alguns critérios de inclusão e exclusão foram elaborados.

Critérios de inclusão:
\begin{itemize}
	\item O trabalho deve ter sido publicado após o ano de xxx.
	
	\item O trabalho deve estar disponível nas bases de dados previamente definidas.
	
	\item O estudo deve ter sido escrito em inglês ou português.
	
\end{itemize}

Como critérios de exclusão, considera-se, além do não obedecimento dos critérios de inclusão, os seguintes critérios:
\begin{itemize}
	\item Trabalho incompleto
\end{itemize}

\section{Condução}

\subsection{Procedimentos de seleção}

\subsection{Seleção de estudos primários}

\subsection{Extração de dados}

\section{Resultado}



 