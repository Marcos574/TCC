\chapter[Metodologia]{Metodologia}

Este capítulo apresenta a metodologia utilizada, que se trata de como o trabalho foi desenvolvido, visando responder as questões elaboradas na introdução.

A metodologia selecionada foi a Revisão Sistemática de Literatura (RSL). Se trata de uma abordagem rigorosa para identificar, avaliar e sintetizar evidências relevantes sobre uma questão específica de pesquisa.
É um método amplamente utilizado em muitas áreas para fixar o entendimento sobre algum tema em específico e direcionar pesquisas futuras. A RSL se trata de um estudo secundário, pois é baseado em estudos primários que já foram publicados.

Para conduzir uma RSL de maneira estruturada e eficiente, o processo é dividido em três fases principais, o \textbf{planejamento}, a \textbf{condução} e os \textbf{resultados}.

\begin{itemize}
	\item \textbf{Planejamento}

A fase de planejamento é crucial para a organização da RSL. O objetivo desta etapa é identificar a necessidade de realizar a revisão e definir de maneira clara os objetivos. Isso inclui duas atividades-chave, sendo elas:

\begin{itemize}
	\item \textbf{Formular as questões de pesquisa}
	
		Funcionam como um guia para o processo de busca e análise.
	
	\item \textbf{Elaboração do protoloco}
	
		 Descreve os critérios de inclusão e exclusão de estudos. Além disso, define quais serão as fontes de dados a serem utilizadas (como bases de dados eletrônicas, livros, anais de conferências e outras publicações relevantes). Por último, mas não menos importante, os procedimentos para coleta e organização das informações. Esse protocolo serve como um guia para toda a RSL.
\end{itemize}

\item \textbf{Condução}

Esta etapa é a fase operativa da RSL, onde a estratégia de busca definida anteriormente é colocada de fato em prática. Essa estratégia pode incluir combinações de palavras-chave, operadores lógicos e filtros específicos que permitam localizar os estudos mais relevantes para a pesquisa nas fontes selecionadas. Uma vez coletados, os estudos passam por uma triagem baseada nos critérios de inclusão e exclusão estabelecidos no protocolo, para assim, filtrar ainda mais os estudos.

Logo após, é realizada uma análise crítica do conteúdo dos estudos selecionados, com o objetivo de extrair e organizar as informações relevantes. A principal finalidade dessa fase é responder às questões de pesquisa propostas, garantindo que a resposta tenha como base as evidências obtidas a partir dos estudos.


\item \textbf{Resultado}

Esta é a fase final da RSL, que é dedicada à documentação e apresentação dos resultados. Aqui, os dados extraídos e analisados são consolidados em um relatório ou artigo científico, que inclui a descrição detalhada dos procedimentos adotados, as evidências encontradas e as respostas às questões de pesquisa. Esta fase também pode envolver a elaboração de tabelas, gráficos e diagramas que auxiliem na compreensão dos resultados.

É importante destacar que os achados da RSL devem ser interpretados considerando-se suas limitações, como possíveis vieses nos estudos primários ou na seleção dos mesmos.

\end{itemize}

\section{Planejamento}

\subsection{Questões de pesquisa}

\subsection{Estratégia de busca}

\subsection{Critérios de seleção}

\section{Condução}

\subsection{Procedimentos de seleção}

\subsection{Seleção de estudos primários}

\subsection{Extração de dados}

\section{Resultado}



 