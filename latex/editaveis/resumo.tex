\begin{resumo}
Este estudo tem como objetivo principal examinar o papel da prototipação no Design de Serviços (DS), explorando as técnicas, ferramentas e processos envolvidos, com foco em sua aplicação no desenvolvimento de soluções centradas no cliente. A pesquisa adotou a metodologia de Revisão Sistemática da Literatura (RSL), estruturada em três fases: Planejamento, Condução e Resultados. Na fase de Planejamento, foram definidas as questões de pesquisa e elaborado o protocolo de revisão. O estudo destaca como a prototipação contribui para a melhoria da experiência do usuário, reduzindo custos e riscos associados ao desenvolvimento de serviços. Além disso, discute os benefícios do design de serviço, diferenciando-o de outras abordagens, como o \textit{design thinking} e a experiência do usuário.

 \vspace{\onelineskip}
    
 \noindent
 \textbf{Palavras-chave}: Design de Serviços, prototipação, Revisão Sistemática de Literatura, técnicas, ferramentas.
\end{resumo}

