\begin{resumo}
Este estudo tem como objetivo principal examinar o papel da prototipação no Design de Serviços (DS), explorando as técnicas, ferramentas e processos envolvidos, com foco em sua aplicação para o desenvolvimento de soluções centradas no cliente. A pesquisa adotou a metodologia de Revisão Sistemática de Literatura (RSL), estruturada em três fases: Planejamento, Condução e Resultados. Na fase de Planejamento, foram definidas as questões a serem respondidas e elaborado o protocolo de revisão. Na fase de Condução, foram identificados e analisados estudos relevantes sobre técnicas e ferramentas de prototipação no DS, utilizando estratégias de busca predefinidas. Por fim, na fase de Resultados, os dados coletados foram sintetizados para responder às questões de pesquisa. Os resultados evidenciaram as principais técnicas e ferramentas de prototipação, destacando suas características, aplicabilidades e impactos no desenvolvimento de serviços. Além disso, foram estabelecidos os pontos de entrada e saída do processo de prototipação, demonstrando sua integração com outras etapas do DS. Concluiu-se que a prototipação é uma etapa essencial no DS, contribuindo para a criação de serviços mais alinhados às expectativas dos clientes, reduzindo custos e aumentando a satisfação do usuário.

 \vspace{\onelineskip}
    
 \noindent
 \textbf{Palavras-chave}: Design de Serviços, prototipação, Revisão Sistemática de Literatura, técnicas, ferramentas.
\end{resumo}

