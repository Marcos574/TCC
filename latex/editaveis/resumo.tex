\begin{resumo}


Na atual economia digital, a qualidade dos serviços e da experiência do cliente é fundamental para a competitividade. O Design de Serviços (DS) oferece abordagens para criar serviços eficazes e centrados nas pessoas, sendo a prototipagem uma prática chave neste processo para tornar ideias tangíveis, testar interações e reduzir incertezas no desenvolvimento.
Este estudo tem como objetivo principal examinar o papel da prototipação no Design de Serviços (DS), explorando as técnicas, ferramentas, processos e caracterizações envolvidas, por meio de uma Revisão Sistemática da Literatura (RSL).
Adotou-se a metodologia RSL, seguindo as fases de Planejamento (definição das questões de pesquisa e protocolo), Condução (busca, seleção e análise de artigos) e Análise dos Resultados da literatura selecionada.
A análise revelou que o DS emprega uma ampla variedade de técnicas de prototipagem (visuais, experienciais, físicas, digitais, estratégicas), aplicadas de forma flexível e iterativa ao longo de diferentes estágios do processo. A prototipagem busca atingir múltiplos objetivos (como comunicação, exploração, teste e empatia) e utiliza diversas ferramentas, desde materiais básicos até tecnologias como IoT e AR/VR, embora a especificação de softwares comuns seja inconsistente nos estudos. O processo transforma entradas (conceitos, requisitos, feedback) em saídas importantes (protótipos, aprendizado, feedback, especificações).
Conclui-se que a prototipagem é uma prática central no DS, facilitando a colaboração e o desenvolvimento de soluções focadas no ser humano.


 \vspace{\onelineskip}
	
 \noindent
 \textbf{Palavras-chave}: Prototipação, Design de Serviços, Revisão Sistemática de Literatura, técnicas, ferramentas.
\end{resumo}

