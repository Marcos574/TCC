\begin{resumo}

Este estudo tem como objetivo principal examinar o papel da prototipação no Design de Serviços (DS), explorando as técnicas, ferramentas, processos e caracterizações envolvidas, por meio de uma Revisão Sistemática da Literatura (RSL). A pesquisa adotou a metodologia RSL, estruturada nas fases de Planejamento (definição das questões de pesquisa e protocolo), Condução (busca, seleção e análise de artigos) e Resultados. A análise revelou que o DS utiliza uma ampla variedade de técnicas de prototipagem (visuais, experienciais, físicas, digitais, estratégicas), aplicadas de forma flexível e iterativa ao longo do processo de design, desde a exploração inicial até a validação final. Essas técnicas são caracterizadas por múltiplos objetivos – como comunicação, aprendizado, teste e empatia – e geram não apenas os protótipos em si, mas também feedback valioso, entendimento aprofundado e conceitos refinados. Uma gama diversificada de ferramentas suporta essas práticas, desde materiais básicos até tecnologias emergentes como IoT e AR/VR, embora a especificação de softwares digitais comuns seja inconsistente na literatura. Conclui-se que a prototipação é uma atividade central e adaptável no DS, essencial para materializar serviços, facilitar a colaboração, testar interações e orientar o desenvolvimento de soluções centradas no ser humano.

 \vspace{\onelineskip}
	
 \noindent
 \textbf{Palavras-chave}: Prototipação, Design de Serviços, Revisão Sistemática de Literatura, técnicas, ferramentas.
\end{resumo}

