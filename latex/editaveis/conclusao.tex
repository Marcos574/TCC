\chapter[Conclusão]{Conclusão}

De maneira geral, a análise confirma que o DS possui uma ampla gama de técnicas de prototipagem disponíveis. Observa-se uma forte presença de métodos visuais e conceituais para lidar com a subjetividade e complexidade dos serviços, bem como um destaque para técnicas experienciais (como role-playing e simulações) que permitem vivenciar e testar as interações humanas fundamentais nos serviços \cite{soto2023prototyping, seko2024transitions}. Além disso, a prototipagem digital e funcional é de suma importância na era dos serviços digitais e conectados (IoT, AR/VR), como apontado por \citeonline{Kim2024, wang2023smartproducts, giraldo2024ecotourism, villa2022integratedcare}. A literatura também aponta para a aplicação da prototipagem não apenas como uma fase final de teste, mas como ferramenta exploratória e de pesquisa contínua ao longo de todo o processo de design \cite{soto2023prototyping, paust2025integrative}. A seleção da técnica depende principalmente dos objetivos específicos, do estágio do processo e do contexto do serviço.

Os objetivos variam da comunicação e exploração inicial à avaliação rigorosa e ao estímulo da empatia. Os métodos são marcados pela iteração, variação de fidelidade, concretização de conceitos, simulação de experiências e co-criação. Já os resultados englobam não apenas os artefatos prototípicos em si, mas também insights importantes, aprendizado organizacional, alinhamento entre \textit{stakeholders} e o refinamento progressivo das soluções de serviço.

Em relação à aplicabilidade das técnicas de prototipagem, a literatura analisada indica que, embora muitos modelos coloquem a prototipagem em fases específicas (normalmente após a ideação e antes da validação final), existe um reconhecimento crescente de sua flexibilidade e aplicabilidade ao longo de todo o processo de DS. A escolha do momento e da técnica de prototipagem está fortemente ligada aos objetivos específicos de cada fase – desde a exploração e empatia iniciais, passando pelo refinamento e comunicação intermediários, até a validação e teste finais. A prototipagem em DS, portanto, não é um evento único, mas um conjunto de práticas adaptáveis usadas iterativamente para dar forma e validar experiências de serviço.

Já em relação às ferramentas, a literatura sugere que o DS utiliza um conjunto variado de ferramentas. Ferramentas básicas de esboço e modelagem física coexistem com softwares de design digital para \textit{mockups} e \textit{wireframes}, ferramentas de hardware para prototipagem IoT, tecnologias imersivas (AR/VR) e equipamentos especializados para simulações experienciais. Ferramentas como Jornada do Usuário, \textit{Wireframe} e \textit{Storyboard} são frequentemente empregadas. A escolha da ferramenta está fortemente ligada à técnica de prototipagem selecionada, ao nível de fidelidade desejado e ao objetivo de comunicação ou de teste.

Levando em conta a comparação entre as ferramentas, a literatura analisada compara ferramentas e técnicas de prototipagem principalmente através das lentes da fidelidade e adequação ao estágio do processo, da funcionalidade principal (comunicação, teste, simulação da experiência) e de fatores contextuais (habilidade do designer, disciplina, colaboração). Não destaca uma única ferramenta ``superior'', mas sim um entendimento de que a escolha deve ser estratégica, alinhando as capacidades da ferramenta (ou técnica) com os objetivos específicos da prototipagem naquele momento do DS. A progressão comum vai de ferramentas de baixa fidelidade para exploração e comunicação inicial para ferramentas de alta fidelidade para teste e validação final, com uma ênfase particular em métodos experienciais no contexto do DS.

Falando sobre entradas e saídas, a prototipagem em DS funciona como um processo de transformação que requer entradas claras (conceitos, compreensão do usuário, requisitos, feedback, intenção, ferramentas) e gera saídas diversas e muito importantes (os protótipos em si, feedback e dados, aprendizado e empatia, conceitos refinados e especificações, resultados organizacionais). Ela materializa o abstrato, permite a exploração e o teste de forma concreta e fornece a base informada para as etapas subsequentes do desenvolvimento do serviço.

