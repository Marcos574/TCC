\chapter[Conclusão]{Conclusão}

Este Trabalho de Conclusão de Curso teve como objetivo investigar o papel e a aplicação da prototipagem no contexto do Design de Serviços (DS) por meio de uma Revisão Sistemática da Literatura (RSL). Buscou-se responder a questões sobre as técnicas, caracterizações, contextos de aplicação, ferramentas, comparações e os processos de entrada e saída da prototipagem em DS, visando consolidar o conhecimento atual sobre esta prática para a inovação em serviços. A análise dos estudos selecionados permitiu visualizar um panorama abrangente, onde os principais achados são sintetizados a seguir.

De maneira geral, a análise confirma que o DS possui uma ampla gama de técnicas de prototipagem disponíveis. Observa-se uma forte presença de métodos visuais e conceituais para lidar com a subjetividade e complexidade dos serviços, bem como um destaque para técnicas experienciais (como role-playing e simulações) que permitem vivenciar e testar as interações humanas fundamentais nos serviços \cite{soto2023prototyping, seko2024transitions}. Além disso, a prototipagem digital e funcional é de suma importância na era dos serviços digitais e conectados (IoT, AR/VR), como apontado por \citeonline{Kim2024, wang2023smartproducts, giraldo2024ecotourism, villa2022integratedcare}. A literatura também aponta para a aplicação da prototipagem não apenas como uma fase final de teste, mas como ferramenta exploratória e de pesquisa contínua ao longo de todo o processo de design \cite{soto2023prototyping, paust2025integrative}. A seleção da técnica depende principalmente dos objetivos específicos, do estágio do processo e do contexto do serviço.

Os objetivos variam da comunicação e exploração inicial à avaliação rigorosa e ao estímulo da empatia. Os métodos são marcados pela iteração, variação de fidelidade, concretização de conceitos, simulação de experiências e co-criação. Já os resultados englobam não apenas os artefatos prototípicos em si, mas também insights importantes, aprendizado organizacional, alinhamento entre \textit{stakeholders} e o refinamento progressivo das soluções de serviço.

Em relação à aplicabilidade das técnicas de prototipagem, a literatura analisada indica que, embora muitos modelos coloquem a prototipagem em fases específicas (normalmente após a ideação e antes da validação final), existe um reconhecimento crescente de sua flexibilidade e aplicabilidade ao longo de todo o processo de DS. A escolha do momento e da técnica de prototipagem está fortemente ligada aos objetivos específicos de cada fase – desde a exploração e empatia iniciais, passando pelo refinamento e comunicação intermediários, até a validação e teste finais. A prototipagem em DS, portanto, não é um evento único, mas um conjunto de práticas adaptáveis usadas iterativamente para dar forma e validar experiências de serviço.

Já em relação às ferramentas, a literatura sugere que o DS utiliza um conjunto variado de ferramentas. Ferramentas básicas de esboço e modelagem física coexistem com softwares de design digital para \textit{mockups} e \textit{wireframes}, ferramentas de hardware para prototipagem IoT, tecnologias imersivas (AR/VR) e equipamentos especializados para simulações experienciais. Ferramentas como Jornada do Usuário, \textit{Wireframe} e \textit{Storyboard} são frequentemente empregadas. A escolha da ferramenta está fortemente ligada à técnica de prototipagem selecionada, ao nível de fidelidade desejado e ao objetivo de comunicação ou de teste.

Levando em conta a comparação entre as ferramentas, a literatura analisada compara ferramentas e técnicas de prototipagem principalmente através das lentes da fidelidade e adequação ao estágio do processo, da funcionalidade principal (comunicação, teste, simulação da experiência) e de fatores contextuais (habilidade do designer, disciplina, colaboração). Não destaca uma única ferramenta ``superior'', mas sim um entendimento de que a escolha deve ser estratégica, alinhando as capacidades da ferramenta (ou técnica) com os objetivos específicos da prototipagem naquele momento do DS. A progressão comum vai de ferramentas de baixa fidelidade para exploração e comunicação inicial para ferramentas de alta fidelidade para teste e validação final, com uma ênfase particular em métodos experienciais no contexto do DS.

Falando sobre entradas e saídas, a prototipagem em DS funciona como um processo de transformação que requer entradas claras (conceitos, compreensão do usuário, requisitos, feedback, intenção, ferramentas) e gera saídas diversas e muito importantes (os protótipos em si, feedback e dados, aprendizado e empatia, conceitos refinados e especificações, resultados organizacionais). Ela materializa o abstrato, permite a exploração e o teste de forma concreta e fornece a base informada para as etapas subsequentes do desenvolvimento do serviço.


\section{Implicações do estudo}

Este estudo traz algumas implicações importantes, tanto para quem estuda quanto para quem pratica o Design de Serviços.


\subsection{Implicações teóricas}

Esta revisão ajuda a organizar o conhecimento atual sobre prototipagem em DS. Ela mostra a grande variedade de técnicas e como a prototipagem vai além do simples teste, sendo usada também para explorar, comunicar e gerar empatia \cite{soto2023prototyping, paust2025integrative}. A flexibilidade da prototipagem, usada em diferentes fases do projeto \cite{paust2025integrative, mager2023product}, sugere que os modelos teóricos sobre processos de DS precisam levar em conta essa adaptabilidade. O estudo também reforça o papel da prototipagem como um passo chave que transforma ideias e informações em aprendizado e decisões concretas. Além disso, mostra a necessidade de a teoria de DS continuar incorporando o uso de novas tecnologias como IoT e AR/VR na prototipagem \cite{Kim2024, giraldo2024ecotourism}.

\subsection{Implicações práticas}

Para os designers de serviço, este trabalho oferece uma lista organizada de técnicas e ferramentas, ajudando a escolher a melhor opção para cada situação. Destaca a importância de usar métodos que simulam a experiência (como \textit{role-playing}) e de incluir os usuários e outros envolvidos no processo \cite{seko2024transitions, asbjornsen2022echange}. Para gestores, a revisão mostra como a prototipagem ajuda a diminuir riscos, incentivar a inovação e alinhar as equipes \cite{paust2025integrative, iriarte2023service}. Entender o que é preciso como entrada para prototipar ajuda a planejar melhor os projetos. Para quem ensina DS, este panorama de técnicas e ferramentas pode ser útil para montar aulas e atividades, mostrando a necessidade de ensinar desde esboços simples até o uso de ferramentas digitais e métodos mais experienciais \cite{Kim2024, mager2023product}.

\section{Limitações da pesquisa}

É importante reconhecer algumas limitações desta Revisão Sistemática da Literatura. A busca por artigos se limitou a bases de dados e termos específicos, o que pode ter deixado de fora estudos relevantes que usam outras palavras ou não estão nessas bases. O foco em artigos revisados por pares, embora aumente a confiança nos resultados, pode excluir informações práticas de outras fontes, como blogs ou relatos de empresas.

A escolha e a análise dos artigos sempre envolvem alguma interpretação do pesquisador. Finalmente, os resultados dependem do que foi publicado pelos autores dos estudos originais; por exemplo, a falta de menção a ferramentas específicas em muitos artigos é uma limitação da própria literatura revisada.

\section{Recomendações para pesquisas futuras}

Com base no que foi encontrado e nas limitações, algumas sugestões para pesquisas futuras são:
\begin{itemize}
	\item Comparar diretamente diferentes técnicas de prototipagem (ex: role-playing vs. mockup digital) para ver qual funciona melhor para cada objetivo (ex: gerar empatia vs. testar usabilidade).
	
	\item Investigar mais a fundo quais ferramentas de software (como Figma, Miro, etc.) e hardware (kits IoT, AR/VR) são realmente usadas na prática do DS e como elas ajudam (ou atrapalham) os designers.
	
	\item Estudar como a prototipagem acontece em setores específicos, como saúde, educação ou serviços públicos, que podem ter desafios diferentes.
	
	\item Criar e testar formas melhores de medir o sucesso de um protótipo, não só se ele funciona, mas também se ele ajudou na comunicação, no aprendizado da equipe ou na empatia.
	
	\item Pesquisar como a cultura da empresa e o apoio da chefia afetam o uso e o sucesso da prototipagem.
	
	\item Explorar como prototipar aspectos mais complexos dos serviços, como a relação entre vários parceiros ou o modelo de negócio como um todo \cite{mager2023product, yan2022pssvalue}.
	
\end{itemize}


\section{Considerações finais}

Este trabalho reuniu informações importantes sobre como a prototipagem é usada no DS. Ficou claro que prototipar é muito mais do que apenas fazer um modelo antes de construir o produto final; é uma parte essencial do processo para entender usuários, testar ideias, comunicar conceitos e dar forma a serviços melhores e mais centrados nas pessoas. Mesmo com as limitações de uma RSL, os resultados mostram a grande variedade de técnicas e ferramentas disponíveis e destacam a flexibilidade e a importância da prototipagem em diferentes momentos do design. Espera-se que esta síntese seja útil para quem estuda e trabalha com Design de Serviços, incentivando o uso mais consciente e eficaz da prototipagem para criar experiências de valor.

