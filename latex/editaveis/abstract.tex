\begin{resumo}[Abstract]
 \begin{otherlanguage*}{english}
 	
This study aims to examine the role of prototyping in Service Design, exploring the techniques, tools, and processes involved, with a focus on its application in the development of customer-centered solutions. The research adopted the Systematic Literature Review methodology, structured into three phases: Planning, Conducting, and Results. In the Planning phase, the research questions were defined, and the review protocol was developed. The study highlights how prototyping contributes to improving the user experience, reducing costs and risks associated with service development. Additionally, it discusses the benefits of service design, distinguishing it from other approaches such as design thinking and user experience. The results demonstrate that prototyping plays a strategic role in the conception of innovative and efficient services, enhancing the user experience and increasing the likelihood of success for the developed solutions.

   \vspace{\onelineskip}
 
   \noindent 
   \textbf{Key-words}: Service Design, prototyping, Systematic Literature Review, techniques, tools.
 \end{otherlanguage*}
\end{resumo}
