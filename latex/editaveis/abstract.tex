\begin{resumo}[Abstract]
 \begin{otherlanguage*}{english}
 	
The main objective of this study is to examine the role of prototyping in Service Design (SD), exploring the techniques, tools and processes involved, with a focus on their application for the development of customer-centered solutions. The research adopted the Systematic Literature Review (SLR) methodology, structured in three phases: Planning, Conducting and Results. In the Planning phase, the questions to be answered were defined and the review protocol was developed. In the Conducting phase, relevant studies on prototyping techniques and tools in SD were identified and analyzed, using predefined search strategies. Finally, in the Results phase, the collected data were synthesized to answer the research questions. The results highlighted the main prototyping techniques and tools, highlighting their characteristics, applicability and impacts on service development. In addition, the entry and exit points of the prototyping process were established, demonstrating its integration with other stages of SD. It was concluded that prototyping is an essential step in DS, contributing to the creation of services more aligned with customer expectations, reducing costs and increasing user satisfaction.

   \vspace{\onelineskip}
 
   \noindent 
   \textbf{Key-words}: Service Design, prototyping, Systematic Literature Review, techniques, tools.
 \end{otherlanguage*}
\end{resumo}
