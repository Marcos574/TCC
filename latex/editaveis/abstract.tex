\begin{resumo}[Abstract]
 \begin{otherlanguage*}{english}
 	
This study aims to examine the role of prototyping in Service Design (SD), exploring the techniques, tools, processes and characterizations involved, through a Systematic Literature Review (SLR). The research adopted the SLR methodology, structured in the Planning (definition of research questions and protocol), Conduct (search, selection and analysis of articles) and Results phases. The analysis revealed that SD uses a wide variety of prototyping techniques (visual, experiential, physical, digital, strategic), applied flexibly and iteratively throughout the design process, from initial exploration to final validation. These techniques are characterized by multiple objectives – such as communication, learning, testing and empathy – and generate not only the prototypes themselves, but also valuable feedback, in-depth understanding and refined concepts. A diverse range of tools support these practices, from basic materials to emerging technologies such as IoT and AR/VR, although the specification of common digital software is inconsistent in the literature. It is concluded that prototyping is a central and adaptable activity in DS, essential to materialize services, facilitate collaboration, test interactions and guide the development of human-centered solutions.

   \vspace{\onelineskip}
 
   \noindent 
   \textbf{Key-words}: Prototyping, Service Design, Systematic Literature Review, techniques, tools.
 \end{otherlanguage*}
\end{resumo}
