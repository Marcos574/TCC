\begin{resumo}[Abstract]
 \begin{otherlanguage*}{english}
 	
In today’s digital economy, the quality of services and customer experience is essential for competitiveness. Service Design (SD) offers approaches to create effective and human-centered services, with prototyping being a key practice in this process to make ideas tangible, test interactions and reduce uncertainty in development.
The main objective of this study is to examine the role of prototyping in Service Design (SD), exploring the techniques, tools, processes and characterizations involved, through a Systematic Literature Review (SLR).
The SLR methodology was adopted, following the phases of Planning (definition of research questions and protocol), Conducting (search, selection and analysis of articles) and Analysis of the Results of the selected literature.
The analysis revealed that SD employs a wide variety of prototyping techniques (visual, experiential, physical, digital, strategic), applied flexibly and iteratively throughout different stages of the process. Prototyping aims to achieve multiple goals (such as communication, exploration, testing, and empathy) and uses a variety of tools, from basic materials to technologies such as IoT and AR/VR, although the specification of common software is inconsistent across studies. The process transforms inputs (concepts, requirements, feedback) into meaningful outputs (prototypes, learning, feedback, specifications).
It is concluded that prototyping is a core practice in DS, facilitating collaboration and the development of human-centered solutions.

   \vspace{\onelineskip}
 
   \noindent 
   \textbf{Key-words}: Prototyping, Service Design, Systematic Literature Review, techniques, tools.
 \end{otherlanguage*}
\end{resumo}
